\documentclass[a4paper,13pt]{article} 
\usepackage{a4,color,graphics,fancyhdr} 
%\usepackage[dvips,colorlinks,bookmarksopen,bookmarksnumbered,citecolor=red,urlcolor=red]{hyperref}


\def\etal{{\it et al.} }
\def\JAS{ {\it J. Atmos. Sci.}, }
\def\MWR{ {\it Mon. Wea. Rev.}, }
\def\QJRMS{ {\it Q. J. R. Meteor. Soc.}, }
\def\hang1{\smallskip\noindent\hangindent=.35truein\hangafter=1}
 
\begin{document} 
%\twocolumn

\title{The Quasi-Balanced Aximetric Evolution model}
\author{Bui Hoang Hai}
\date{\today}
\maketitle 

\section{Model Description}
\label{sec:ModelDescription}

\subsection{The Equations}
\label{sec:Equations}

The tendency equation for tangential wind speed in $r-z$ coordinates is:
\begin{equation} 
\frac{\partial v}{{\partial t}} = -u\frac{\partial v}{{\partial r}} - w\frac{\partial v}{{\partial z}}
	                                     -\frac{uv}{{\partial r}} -fu + \dot{V} 
\label{eq_tendency}
\end{equation}

$\dot{V}$ is the momentum source due to eddy diffusion has the form:

\[{K_H}\left( {{\nabla ^2} - \frac{1}{{{r^2}}}} \right)v + \frac{{\partial \tau }}{{\partial z}}\]

where $K_H$ is the horizontal eddy diffusion coeficient, $K_v$ is the vertical eddy diffusion coeficient 
${\nabla ^2} = \frac{{{\partial ^2}}}{{\partial {r^2}}} + \frac{1}{r}\frac{\partial }{{\partial r}}$ and
$\tau  = -{K_v}\frac{{\partial v}}{{\partial z}}$ is vertical eddy flux.
  

The Sawyer-Elliassen (SE) equation in radial-height coordinates has the form: 

\[ \frac{\partial }{{\partial r}}\left[ {-g\frac{{\partial \chi }}{{\partial z}}\frac{1}{\rho r}\frac{{\partial \psi }}{{\partial
r}} - \frac{\partial }{{\partial z}}(\chi C)\frac{1}{\rho r}\frac{{\partial \psi }}{{\partial z}}} \right] + \]
\[ \frac{\partial }{{\partial z}}\left[ {\left( {\xi \chi (\zeta + f) + C\frac{{\partial \chi }}{{\partial r}}}
\right)\frac{1}{\rho r}\frac{{\partial \psi }}{{\partial z}} - \frac{\partial }{{\partial z}}(\chi C)\frac{1}{\rho
r}\frac{{\partial \psi }}{{\partial r}}} \right] = \]
\begin{equation}
g\frac{{\partial}}{{\partial r}}\left({\chi^2 \dot \theta}\right) + \frac{\partial }{{\partial z}}\left(C{\chi^2 \dot \theta}\right)
+ g\frac{{\partial F_\lambda}}{{\partial r}} + \frac{\partial }{{\partial z}}\left(CF_\lambda\right)
\label{eq_se}
\end{equation}

where $\chi = 1/\theta$, $C=v^2/r+fv$, $\xi = 2v/r + f$, $\zeta = (1/r)(\partial (rv)/\partial r)$ is the vertical component of relative vorticity; $\eta=\zeta+f$ is the absolute vorticity; $\dot{\theta}=d\theta/dt$ is diabatic heating rate.
$\psi$ is a stream function that satisfies 
	\[u=-\frac{1}{r\rho} \frac{ \partial \psi}{ \partial z},
	w=\frac{1}{r\rho} \frac{ \partial \psi}{ \partial r}
\]

The heating source can be defined as a function of r and z as:
 
\begin{equation}
	\dot{\theta}=M\cos{(\pi \delta_r/W)} \cos{(\pi \delta_z/H)}
\label{eq_thetadot}
\end{equation}
Where $M$, $W$, $H$ is the magnitude, width and height of the source accordingly. 
$\delta_r$, $\delta_z$ is the relative distance is the heating center. Note that in this
formula, a potential radius $R$ is used such as:

	\[1/2 fR^2 = rv + 1/2 fr^2
\]

For a prescribed tangential wind distribution $v(r,z)$ and environment field of pressure and temperature, a complete
balanced fields of a TC can be calculated using a unapproximated method of Smith (2006). After a diabatic heating source is defined by (\ref{eq_thetadot})in potential radius coordinates and then transformed into physical coordinates, a solution for the toroidals stream function $\psi$ and hence radial and vertical wind speed can be obtained by solving the SE equation (\ref{eq_se}) using an over-successive relaxation similar to Bui \etal{}(2009) . Then, equation (\ref{eq_tendency}) can be integrated to obtain tangential wind speed in the next timestep. After each timestep, potential temperature and
density can be {\it re-balanced} using the method of Smith (2006).

\subsection{Numerical Methods}
\label{sec:NumericalMethods}

The over-successive relaxation method is the same as the one used in Bui \etal (2009).
The diffusion terms are diffentiated as:
	\[D_z(k) = \frac{K_z}{\Delta z^2} ( v_{k+1} +  v_{k-1} - 2v_k)
\]
	\[D_r(i) =  \frac{K_r}{\Delta r^2} (v_{i+1} +  v_{i-1} - 2v_i) + \frac{K_r}{2r_i\Delta r} (v_{i+1} -  v_{i-1} ) 
\] 
where $i$ and $k$ are the radial and vertical gridpoint number accordingly.

%\newpage
%\thispagestyle{empty}
%\addcontentsline{toc}{chapter}{\numberline{References}{.}}
%\vspace*{.5in}
\begin{center}
{\bf References}
\end{center}


\hang1 {H. H. Bui, R. K. Smith, M. T. Montgomery, and J. Peng, 2009: Balanced and unbalanced aspects of tropical-cyclone intensification. {\QJRMS} {\bf 135}, 1715-1731}.

\hang1 {Smith RK. 2006: Accurate determination of a balanced axisymmetric vortex. {\it Tellus}, {\bf 58A}, 98-103.

\end{document}
